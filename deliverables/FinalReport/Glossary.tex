\newglossaryentry{payload}{
	name=payload,
	description={A subsection of a rocket that is not essential to the rocket's 
		operation. A payload is placed in a can, mounted on a standard base plate. 
		A payload completes some specific task.}
}

\newglossaryentry{can}{
	name=can,
	description={A can is a segment of the rocket in which payloads can be 
		placed. A can constitutes a standard length of rocket, defined by the 
		RockSat-X program.}
}

\newglossaryentry{Arm Assembly}{
	name={Arm Assembly},
	description={The Hephaestus Arm Assembly includes the arm, the rotating arm
		base, the camera, and base. It is the portion of the payload that is 
		extended during Science mode.}
}

\newglossaryentry{api}{
	name=API,
	description={Application Programming Interface. The set of functions and 
		classes that a given library exposes for other programs to make use of its
		provided functionality.}
}

\newglossaryentry{ascii}{
	name=ASCII,
	description={American Standard Code for Information Interchange. Each 
		alphabetic, numeric, or special character is represented with a 7-bit 
		binary number. 128 possible characters are defined.}
}

\newglossaryentry{port}{
	name=port,
	description={To transfer software from one system or machine to another.}
}

\newglossaryentry{ter}{
	name=TE-R,
	description={The TE-R line is a redundant electrical input to the system
		that enables at a predetermined time during flight.}
}

\newglossaryentry{te1}{
	name=TE-1,
	description={The TE-1 line is an electrical input to the system that 
		enables at a predetermined time during flight.}
}

\newglossaryentry{replay}{
	name=replay,
	description={To replay a dataset is to reproduce preexisting data in a manner that
		simulates how it was generated, e.g. output the data in the same timeline that each
		data point was generated.}
}

\newglossaryentry{plot}{
	name=plot,
	description={An interactive window generated by the Python library matplotlib to
		display some dataset on an x and y axis.}
}

\newglossaryentry{apogee}{
	name=apogee,
	description={The point at which the rocket has finished its ascent and payloads
		are allowed to deploy.}
}

\newglossaryentry{matplotlib}{
	name=matplotlib,
	description={A Python library for drawing and manipulating graphs.}
}

\newglossaryentry{binstring}{
	name=binary string,
	description={An ordered sequence of 1's and 0's.}
}

\newglossaryentry{deployable}{
	name=deployable,
	description={Any portion of the payload that is expanded from its original 
		configuration once in a space-like environment.}
}

\newglossaryentry{npm}{
	name=npm,
	description={npm is a package manager for NodeJS, a javascript library. It
		is used to install various `npm' packages for NodeJS servers.}
}

\newglossaryentry{cspace}{
	name=configuration space,
	description={The Configuration Space (or C-Space) is a 4 dimensional space with
		a mapping to 3D real space that is used to represent the possible 
		configurations of the arm's motors.}
% Removed the following lines because the definition is not the appropriate
% place to explain implementation details.
% The C-Space is stored in a 4D array of 
%characters. Possible valid configurations are marked as such in the C-Space and 
%this data is used to plot the arm's path.}
}

\newglossaryentry{dof}{
	name=degree-of-freedom,
	description={The directions in which independent motion can occur. In the case of the Hephaestus arm, there are four degrees of freedom.}
}

\newglossaryentry{obc}{
	name={on-board computer},
	description={The microcontroller responsible for the control and experiment systems on the payload.}
}

\newglossaryentry{microgravity}{
	name=microgravity,
	description={An environment where the effect of gravity is significantly less than earth's.}
}

\newglossaryentry{eeprom}{
	name=EEPROM,
	description={Electrically erasable programmable read-only memory. A type of	
		persistent storage for small amounts of data.}
}

\newacronym{psu}{PSU}{Portland State University}
\newacronym{psas}{PSAS}{Portland State Aerospace Society}
\newacronym{gui}{GUI}{Graphical User Interface}
\newacronym{wff}{WFF}{Wallops Flight Facility}
\newacronym{osu}{OSU}{Oregon State University}
\newacronym{OBC}{OBC}{On-Board Computer}
\newacronym{aiaa}{AIAA}{American Institute of Aeronautics and Astronautics}