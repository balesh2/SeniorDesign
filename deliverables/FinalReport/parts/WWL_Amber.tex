\subsubsection{Technical Information}
During the course of this project, I had the honor of adapating new technical abilities. Prior to this project,
I had never worked with any microcontrollers, as I am an HCI-applied option computer science major and am not required to take
ECE 375. Thus, when we were developing code for the ATMega128, a lot of the architecture and developmental practices
were completely new to me. While I was very familiar with C, I had never written code that actually control an embedded system.
Gaining familiarly with the different registers and ports in order to write our project was helped by the electrical engineers,
 but when I was doing the
SD Card development, I was almost entirely alone in trying to follow the order of calls through a 6,000+ line file of complex C code.
So the act of syntesizing my existing programming knowledge with this new, foreign ATMega128-specific code lead to me honing my existing
programs more but also allowing me to strengthen my ability to adapt to confusing situations and gain a better top-level down
understanding of other people's code. After working on this project, I feel more confident in my ability to write C code and 
to understand and evaluate other code given to me.

\subsubsection{Non-Technical Information}
On top of just technical skills, I have also gained non-technical skills. This project has allowed me to learn how to communicate
better with people outside of my major as I frequently had to work with the mechanical engineers and electrical engineers who do not
have the same level of domain knowledge when it comes to programming. They, however, are more knowledgeable in their respective domains
so ensuring that we were able to simplify complex topics to a point that we both understand while not losing any of the important
intricacies or nuances too much was an important skill that I feel I enhanced over the course of the term. In the beginning of the 
project, I feel our team had many more miscoummunications than in the later parts of the project.

\subsubsection{Project Work Information}
Most of the information I learned from working in this project, I had learned to a much smaller degree in some of my previous group-work
ventures. However, this was the first time I had done work an actual client so that made the dynamic slightly different as every change
we made to the project structure or large implementation efforts needed to be reviewed by her. Nancy was a great client, though, and
was consistently supportive and impressed with the work we had accomplished. I also learned a fair deal about the large life cycle
these projects go through. I have never attempted to send anything into space (not surprising), but it takes a long time and a lot of
effort, both on our part and our sponsors part, to organize and strategize how to get multiple teams from universities all across
the country to be ready and able to send whatever their project is into space. I was consistently impressed with how professional
and knowledgeable our contact in Colorado for the RockSat-X program was both on our project and all the various components
that needed to be function correctly in order for our payload to be onboarded and sent on the ship. I never knew just
how much needed to be done in order to get these payloads onto a ship.

\subsubsection{Project Management Information}
In terms of project management, I learned how important it is to have consistent meetings and smaller, sub-team meetings.
During the most intense development period, winter term, our group splintered into sub-teams that were required to
meet once a week to discuss work that had been done cross-functionally (such as one mechanical engineer, one
electrical engineer, and one computer science team member all meet) to discuss what work has been done
on their end of the project and whether they needed anything from one of these other disciplines. The teams
also had "goals" in place to ensure that all the parts that required knowledge from all three groups
would get completed. For example, I was part of the "Structures" group that had the goal of making
the hardware of the arm with the electrical engineer's motors were all wired and able to move freely without
getting caught. It was important to have all discplines represented within this group as all of us had
different requirements that we were more knowledgeable about (for example, I was more familiar with
the pathing through space we wanted the arm to achieve). I had never worked in smaller groups
like this that had such focused goals, but I really enjoy this method of breaking a large group
into focused sub-teams to ensure that all the moving parts of a project are not only done, but done
with respect to all the other parts of the larger project. This was crucial to the success of a project like ours that is
cross-discplinary. However, I could see it working in some other settings I've been in such as my research group
where there are lots of us all working on different projects but all need to come together to work on one final paper, for example.

\subsubsection{Team Work Information}
I learned a lot about team work through this project. The cross-discplinary aspect allowed me to learn better how to work within
a larger team, as discussed in "Project Management Information". I also learned a lot about my fellow computer science teammates during
this project, though. I learned how important clear and effective communication is, especially when things start crumbling.
My team and I had a difficult time during early winter term due to teammates not clearly communicating whether they were doing
the work assigned. We also had difficulties involving work being done far later than expected and series of excuses as to why these things
may be late. I learned that I think I am a bit too easily persuaded by these issues. Instead of forcing my other teammates to do the work
assigned to them, I normally just gritted my teeth and did the extra work given to me. However, in my future endeavors, I want a more
respectful two-way street when it comes to work promised and work done. I want to work with people that say they will do the work, and 
actually do it. It is very stressful when this is not the case as I have experienced first hand, not only in this class but in other
team projects as well, but I now am more resolved to not allow for this treatment any longer. While this all sounds very negative,
for the most part, I really enjoyed my team and am proud of the work we accomplished. I just wish it wouldn't have been so stressful
at some points, and I feel I could've mitigated some of these issues.

\subsubsection{If you could do it all over, what would you do differently?}
As just mentioned in "Team Work Information", I would be more assertive when discussing how important getting the work done on
this project is to me. I would advocate more for better communication practices earlier on. I believe a lot of the issues
I have already mentioned could've been avoided with these measures put in place. Beyond these complains, I wouldn't change anything.
I'm really happy with this project and the work we've done. Seeing that beautiful arm move and getting to say I sent it into space: I
wouldn't change that at all.

