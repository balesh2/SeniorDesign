\subsubsection{Week 1}
\subsubsubsection{Helena Bales}
\textbf{Progress} \\
This week was a light one for Senior Design. I attended the first class on Wednesday and met with my group and the TA on Tuesday. Both meetings gave me an overview of the deadlines and expectations for this term. I met with the whole RockSat-X group on Wednesday for a meeting, pizza, and games. The other team members kept calling me Amber. We did not have a pathing and automation cross-functional group meeting this week since we didn't figure out scheduling soon enough. I did some work on the Parser for the C-Space.

\textbf{Plans} \\ 
I plan to finish the C-Space parser next week and move on to pathfinding. I will also be making some funding requests so that we can hopefully travel to the integration testing and launch.

\textbf{Problems} \\ 
The main problem for this week is a funding shortage that may mean that we will not be able to send everyone that we need to the integration testing and launch at Wallops.

\subsubsubsection{Amber Horvath}
\subsubsubsection{Michael Humphrey}
\textbf{Progress} \\
No progress this week.

\textbf{Plans} \\
Plan for next week is to implement more of the telemetry interface.

\textbf{Problems} \\
No problems.

\subsubsection{Week 2}
\subsubsubsection{Helena Bales}
\textbf{Progress} \\
This week I finished the parser for the C-Space. It reads the configuration space data from a file. The data is in the form of four angles separated by a semicolon from the next set of angles. The parser reads these angles and uses floor to convert them to integers then marks that location in a 37x37x37x37 array as a 0, indicating that it is a valid configuration of motors, or not blocked. There were no required attendance Senior Design classes this week. I had a TA meeting on Tuesday and an all-team meeting on Wednesday. I also had a pathing and automation meeting on thursday, but due to miscommunications, only one other person showed up. It was disappointing that no one else came but it gave me the chance to get Subret caught up on what we were talking about since he replaced Ian (who graduated) as the ME rep for Pathing. In explaining the C-Space to him I realized that I missed a bug when writing my parser in that I forgot to convert negative degrees to their positive equivalents before putting them in the array.

\textbf{Plans} \\
Next week will involve finishing up the second draft of the RockSat-X poster and continuing work on the pathing and automation tasks. I developed a plan for the pathing and automation tasks during our meeting this week. Next week I will be fixing that bug in the parser, delegating a C-Space visualization in Matlab to James, and starting the Pathfinding portion of the code.

\textbf{Problems} \\
The problems that I am currently facing are in receiving enough support from the group at large for my pathfinding and automation tasks. I think this will improve next week as our schedules all become more normal.

\subsubsubsection{Amber Horvath}
\subsubsubsection{Michael Humphrey}
\textbf{Progress} \\
No progress this week.

\textbf{Plans} \\
Finish implementing telemetry.

\textbf{Problems} \\
No blockers.

\subsubsection{Week 3}
\subsubsubsection{Helena Bales}
\textbf{Progress} \\ 
This week I delegated making a visualization of the configuration space to James. I had trouble explaining what I needed and what to do because we were not able to meet in person. That resulted in some frustration, but I think he will be able to get the visualization done. I also spent a lot of time on the poster this week. There were a lot of tiny edits to be made in the formatting and content. I also took another group picture of all of the Software Team and some individual photos of everyone for them to use on other things.

\textbf{Plans} \\ 
Next week will be continuing work on the pathing and automation tasks. I will also continue working with James on the visualization of the C-Space.

\textbf{Problems} \\
I am currently having problems communicating how to make a visualization of a 4D array with James over text.

\subsubsubsection{Amber Horvath}
\subsubsubsection{Michael Humphrey}
\textbf{Progress} \\
This week, in light of the SD card still being unfinished, I implemented logging to the eeprom on the microcontroller. The eeprom only has 4 kilobytes of available memory, so it is only a mediocre replacement. We no longer have time to fix the SD card library we are using, so we must pursue other means of telemetry. Telemetry via the provided telemetry pins is fully implemented. Telemetry via SD card is partially implemented, but cannot be finished if the SD card library can't be fixed. Telemetry via eeprom is fully implemented. Until SD card is unblocked, the telemetry component is complete.

\textbf{Plans} \\
Last week, we had a team meeting. We have further split up the work, and my part is to meet with the instructors to review deadlines. I also plan on getting the poster completed and submitted. I will also be taking control of writing a readme for the github repo and copy test code into science mode file.

\textbf{Problems} \\
No problems.

\subsubsection{Week 4}
\subsubsubsection{Helena Bales}
\textbf{Progress} \\ 
This week I was able to get the final visualization from James. I was able to answer the last few questions he had, and he produced a good quality visualization for us to include on our poster. I also took updated group pictures for us as well as some individual pictures for everyone to use on their linkedin profiles. I once again continued working on Pathing and Automation tasks. Finally, I continued to make some changes to the Final Expo Poster and got it approved by McGrath and Dr. Squires.

\textbf{Plans} \\ 
On Monday I will submit the final expo poster for printing and turn in the photo release form. Next week I will interview Evan on Thursday and write the WIRED-style review of his project. As always, I will be continuing my work on the Pathing and Automation code.

\textbf{Problems} \\ 
I am having trouble getting support from the Electrical Engineers. They are also very busy with work and classes.

\subsubsubsection{Amber Horvath}
\subsubsubsection{Michael Humphrey}
\textbf{Progress} \\
This week I am making sure me and my team is ready for expo. I am following up with important people and assignments to make sure everything is complete and deadlines are being met. This includes but is not limited to making finals edits on our poster, getting out client's signature for the poster, submitting the poster, and making sure everyone has signed the necessary forms for expo.

\textbf{Plans} \\
I have completed both the video and telemetry components. I have only the data processing component left to implement as my portion of the project, but that is currently blocked until we receive further instruction from NASA. I have nothing left to implement for this project. I will spend the next week getting ready for expo and working on my presentation.

\textbf{Problems} \\
I am waiting for after Integration testing with Wallops and we receive the format of telemetry data. Then the implementation of the data processing component can begin. I will not be directly working on this component, but rather guiding James on implementing the component.

\subsubsection{Week 5}
\subsubsubsection{Helena Bales}
\textbf{Progress} \\ 
This week I submitted our final poster for printing and turned in my photo release form to the Engineering building. I interviewed Evan on Friday and wrote my WIRED-style review. I continued working on my pathing and automation code, and finished the code that makes the motors step through a path.

\textbf{Plans} \\ 
Next week I will be finishing the Pathing and Automation code. I will continue to try to work with the Electrical Engineers and continue to try to secure funding for our travels. I will also start working on my slides for the Progress update.

\textbf{Problems} \\ 
I am still having trouble getting the support I need from the other engineers. Hopefully things improve next week.

\subsubsubsection{Amber Horvath}
\subsubsubsection{Michael Humphrey}
\textbf{Progress} \\
This week I made some minor changes to the telemetry code.

\textbf{Plans} \\
Next week I plan to support Amber and Helena with development of the rest of the system.

\textbf{Problems} \\
No blockers.

\subsubsection{Week 6}
\subsubsubsection{Helena Bales}
\textbf{Progress} \\ 
This week I attempted to finish the pathing and automation code. I ran into some difficulties with this that will require me to finish the code next week. I was introduced to an OSU Foundation grant writer so this week I emailed with him about potential travel grants for us to apply to. He found one potential one, but it requires that someone on the team be under 21, and none of us are. He will still be able to set up a crowdfunding campaign through OSU for us. Other than that, I started working on my slides for the Midterm Progress Update due next week.

\textbf{Plans} \\ 
Next week I will be finishing up the code for the Pathing and Automation, as well as my slides for the progress update, the video of the progress update, the FSMR presentation for RockSat-X, setting up the crowdfunding campaign through OSU, creating print visuals in addition to our poster for Engineering Expo, submitting security clearance and background check authorization forms to Wallops, doing the Poster Extra Credit thing on Monday at 4 with Kirsten, and Expo on Friday.

\textbf{Problems} \\ 
Not enough hours in the day. Also my computer has some issues with AVR so I have to find another computer to use. I have also had a hard time getting in contact with the Electrical Engineer who can help me with the Program Memory part of my code, which is something that I am not particularly skilled or practiced at, so I could really use the help.

\subsubsubsection{Amber Horvath}
\subsubsubsection{Michael Humphrey}
\textbf{Progress} \\
This week I made some more minor tweaks to the telemetry code. I also sat down with Amber for 2 hours and worked on making the code in the repo to compiler. We also added a bunch of missing code in the off and retract phases. We also refactored a bunch of code to make it more readable.

\textbf{Plans} \\
Next week will be spent waiting for Helena to finish writing the code for the science phase.

\textbf{Problems} \\
Blocking on science phase being finished.

\subsubsection{Week 7}
\subsubsubsection{Helena Bales}
\textbf{Progress} \\ 
\textbf{Plans} \\ 
\textbf{Problems} \\ 
\subsubsubsection{Amber Horvath}
\subsubsubsection{Michael Humphrey}
Undoubtedly I have learned more from this class in the past year then I have learned from any other class or life experience. Working with a team of twelve engineers with a common goal is a task that is almost certainly destined for failure, but in the case of this project it was a spectacular experience. Each subteam—two mechanical engineering, one electrical engineering, and one computer science team—was a pleasure to work with. Typically collaborations with this many people are bogged down with red tape and communication mishaps, but we were able to all work together to get work done. Each team was responsible for a specific portion of the project, but made sure that all other teams were on board with any major decisions. Even though we as the CS team didn't care how the physical payload was designed, both ME teams made sure to include us on all decisions that were made. The EE team was especially helpful in making sure we had all the drivers and tools necessary to develop the code for the payload. I can only hope that my future coworkers will be as considerate and helpful as my team members were on this project.

Additionally, I loved working with physical hardware for this project. A lot of times in CS when we work with "hardware," it's really just interfacing with a well-defined API for some connected device. On the contrary, this project required us to write and debug embedded code for a microcontroller. Working with embedded code made me long for the days when I would get unexplained seg faults, and made me realize I was downright spoiled when working with high-level languages like Python. But the satisfaction of watching a device operate from the code I had birthed with my own fingers is unmatched. It's just not the same as seeing the terminal output of an algorithm or getting a message box to pop up from a GUI. Watching the arm operate according to our specification (and only occasionally acting homicidally) is one my favorite moments of this project.

This project has given me a lot of perspective that I had not gotten in any of my classes in my academic career. I have learned a lot from my two CS teammates about what it means to work in a team. I have learned a lot from my ME and EE teammates about what it means to collaborate from different disciplines on a project. I have learned what it takes to interface with outside vendors to get parts. I have learned what it takes to raise funds for a project. I have learned what it takes to meet deadlines and satisfy a client's requirements. I have learned how to write documentation for a project. All of these skills have prepared me to fulfill my potential as an employee in the workforce.

\subsubsection{Week 8}
\subsubsubsection{Helena Bales}
\textbf{If you were to redo the project from Fall term, what would you tell yourself?} \\ 
If I were to redo the project from Fall term I would have us divide up the work differently. I would also want to be better about creating timelines for the whole team (including the ME's and EE's) so that we could understand our dependencies better from the beginning. We were making a lot of it up as we went along since OSU hasn't ever made a rocketry payload for space. I also would have requested that we remove one degree of freedom from the arm since it would have made things a lot easier, but the challenge has been fun.

\textbf{What's the biggest skill you've learned?} \\ 
I have learned a lot about configuration spaces, pathfinding, and have been able to wrap my head around 4D arrays. I have simultaneously been learning about Machine Learning (I wonder how an AI would feel about us learning to make them learn), so I have been thinking about how to incorporate some ML into this project in the future, especially in reacting to changing environments.

\textbf{What skills do you see yourself using in the future?} \\ 
I think that I will use my space robotics skills in the future. It's a pretty niche skill, but I think Intelligence and Space Research might be the right place to use that. I will, if nothing else, continue to use my embedded C skills to do hardware projects.

\textbf{What did you like about the project, and what did you not?} \\ 
I liked getting to think about pathfinding and 4D arrays, but I did not like dealing with data storage or telemetry, so I was pretty happy when Michael and Amber took the lead on those two things. I did not enjoy working with so many other engineers at first as it could be pretty frustrating when we didn't all know each other very well, but now that is one of my favorite things about it, because we are a team.

\textbf{If your project were to be continued next year, what do you think needs to be working on?} \\ 
I hope that this project will continue next year. There are a lot of improvements to be made. I would like to see the following improvements (in order):

\begin{enumerate}
\item{Dynamic pathfinding on orbit - requires more processing power or longer experiment duration or both}
\item{Dynamic targeting on orbit - requires 1.}
\item{Dynamic CSpace Mapping on orbit - requires data from multiple angles and Inverse Kinematics. requires 1.}
\item{A second arm - requires 1, 2, 3, and a longer mission, probably on a satellite}
\item{Tool use with one arm}
\item{Tool use with two arms interacting with each other}
\item{Image processing and object recognition}
\item{...}
\end{enumerate}
I could detail my whole idea for construction in space, but I think that is enough for one night.

\subsubsubsection{Amber Horvath}
