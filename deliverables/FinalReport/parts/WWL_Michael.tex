\subsubsection{Technical Information}
Through this project, I was able to expand my technical knowledge in many
different ways.
From my knowledge of C/C++, I learned how to program in embedded C.
Although it is the same language as vanilla C, there are many subtle but
important differences that a programmer needs to know.
I learned many of these tips and tricks from members of the Electronics team
who have worked extensively with embedded C.
Additionally, I used my knowledge of programming the ATmega128 chip in assembly
from ECE 375 to understand how to program the same chip in C.
While programming in C provides the familiar constructs such as conditionals
and loops, I still needed to interact with the hardware through specific bits
in various registers.
I was able to build on what I knew from ECE 375 to accelerate my learning with 
interacting with the ATmega128 chip in C.

\subsubsection{Non-Technical Information}
I learned a lot about robots, robotic movements, and spacecraft over the course
of the past year.
This is the first project I've worked on that can interact with its environment
outside the context of a computer.
It brings in a whole lot of complexity dealing with physics, inverse kinematics,
and other properties of real-world environments in comparison to the
mathematically well-defined environment that computer programs operate in 
virtually.
It has exposed me to the reality that software exists to solve real-world
problems, not abstract problems commonly in academia.

\subsubsection{Project Work Information}
I learned a lot this year working on a project of this scale.
It is a great experience to see many different parts of the project made by
many different people come together to accomplish a single goal.
That being said, I think there were a lot of shortcomings as well.
Each of the CS team members took ownership of a specific portion of the code.
This helped reduce version control conflicts as well as made each of us
expert in that area of the code.
However, documentation was definitely lacking in all areas.
It is highly unlikely that one team member could quickly and efficiently fix a
bug in another team member's code.


\subsubsection{Project Management Information}
I have learned that project management is an important part of the success of
the product.
Without proper management, deadlines can slip by unnoticed and implementation
can stall with petty squabbles early in the design phase.
It takes an excellent project manager to keep the project moving in a positive
direction and everyone working together productively.
I have learned to identify weaknesses in my team and step up my duties to that
of a product manager when I notice things aren't happening as they need to be.

\subsubsection{Team Work Information}
I have learned a tremendous amount about working with other engineers on a
project.
I learned there needs to be a huge amount of coordination between engineering
disciplines in order to minimize time wasted.
The CS team depended on the designs of the EE team, which in turn depended on
the designs of the ME team.
As it were, the CS team had the least amount of time to accomplish our tasks
because we had to wait for the EEs to finalize their design.
There were many ways this could have been parallelized so there was less time
wasted for other teams.

On the other hand, there was a large amount of coordination between engineering
disciplines on the design of the \gls{payload}.
Every team had input at each stage of the design.
The amount of collaboration was impressive, and it helped to mitigate issues
that may have otherwise gone unnoticed until implementation.

\subsubsection{If you could do it all over what would you do differently?}
If I had to do this project over again, I wouldn't change a lot.
I would try to parallelize the implementation of the different parts of the 
\gls{payload} (i.e. the mechanical, electrical, and software components).
I would also work to mitigate the interpersonal difficulties that came up in
the various groups throughout the year.
