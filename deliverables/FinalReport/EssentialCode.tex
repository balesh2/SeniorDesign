\subsection{Pre-Processing}
\subsubsection{CSpace\_Mapping.ino}
\begin{lstlisting}[language=C]
#include <stdio.h>

//=========DEFINITIONS==================
int encA_pin1 = 6;
int encB_pin1 = 7;
long encA_ticks1 = 0;
//for polling implementation
int n = LOW;
int encA_pin1_prev = LOW;

unsigned long time;
char buffer [50];
int i = 0;

int encA_pin2 = 2;
int encB_pin2 = 4;
volatile long encA_ticks2 = 0;

int encA_pin3 = 3;
int encB_pin3 = 5;
volatile long encA_ticks3 = 0;

const float pi = 3.14;

volatile int state = HIGH;

volatile float encA_degree1 = 0;
volatile float encA_degree2 = 0;
volatile float encA_degree3 = 0;

/*
2 - Motor 1: A
3 - Motor 1: B
4 - Motor 2: A
5 - Motor 2: B
6 - Motor 3: A
7 - Motor 3: B


*/
//=========SET-UP==================
void setup() {
  // put your setup code here, to run once:
pinMode(encA_pin1, INPUT);
pinMode(encA_pin2, INPUT);
pinMode(encA_pin3, INPUT);

digitalWrite(encA_pin1, HIGH);
digitalWrite(encA_pin2, HIGH);
digitalWrite(encA_pin3, HIGH);

pinMode(13, OUTPUT);

// INIT interrups

//switch motor 1 to polling
//attachInterrupt(digitalPinToInterrupt(encA_pin1),encoderA1,RISING);
attachInterrupt(digitalPinToInterrupt(encA_pin2),encoderA2,RISING);
attachInterrupt(digitalPinToInterrupt(encA_pin3),encoderA3,RISING);

Serial.begin(9600);
}


//========= LOOP ==================
void loop() {
  // put your main code here, to run repeatedly:

  time = millis();

  //Poll innermost motor to circumvent limited number
  //of Arduino UNO interupts
  n = digitalRead(encA_pin1);
   if ((encA_pin1_prev == LOW) && (n == HIGH)) {
     if (digitalRead(encB_pin1) == LOW) {
       encA_ticks1 -= 1.0;
     } else {
       encA_ticks1 += 1.0;
     }
   } 
   encA_pin1_prev = n;

  digitalWrite(13,state);

  //if enough time has passed, print a status update
  //print status update every .5 seconds

  //output the valid configurations to Serial which will then be stored in a file
  encA_degree1 = encA_ticks1*2.25;
  encA_degree2 = encA_ticks2*2.25;
  encA_degree3 = encA_ticks3*2.25;

  Serial.println("0");
  Serial.println(encA_degree1);
  Serial.println(encA_degree2);
  Serial.println(encA_degree3);
  Serial.println(";");

  delay(500);
}

//Arduino UNO only has two interupts
//Switch innermost motor to polling

//void encoderA1(){
//  state=!state;
//  if(digitalRead(encB_pin1)==HIGH){
//    encA_ticks1=encA_ticks1+1.0;
//  }else{
//    encA_ticks1=encA_ticks1-1.0;
//  }
//  }

void encoderA2(){
  state=!state;
  if(digitalRead(encB_pin2)==HIGH){
    encA_ticks2=encA_ticks2+1.0;
  }else{
    encA_ticks2=encA_ticks2-1.0;
  }
}
  
void encoderA3(){
  state=!state;
  if(digitalRead(encB_pin3)==HIGH){
    encA_ticks3=encA_ticks3+1.0;
  }else{
    encA_ticks3=encA_ticks3-1.0;
  }
  }

\end{lstlisting}
\subsubsection{parser.cpp}
\begin{lstlisting}[language=C++]
#include <stdio.h>
#include <stdlib.h>
#include <iostream>
#include <fstream>
#include <string>
#include <cstring>
#include <time.h>
#include <unistd.h>
#include <math.h>

using namespace std;

/************************************************************************************************
 * Program: C-Space Data Parser
 * Author: Helena Bales
 * Date: April 12th, 2017
 * Description: A C++ implementation of a parser for the Hephaestus Configuration Space data. The 
 * 	parser creates a 4-D array of boolean values with a 0 representing a valid configuration 
 * 	and a 1 representing an invalid configuration.
 * Input: ./test1/data1 - Hephaestus C-Space data as defined in project docs
 * Output: ./cspace.out - Output file containing boolean data from 37x37x37x37 array
 ***********************************************************************************************/

int main() {
	char* myFile;
	ifstream sourceFile;
	ofstream outfile;
	string line;

	int accuracyFactor = 10;
	int MAX_ANGLE = 360;
	int x = MAX_ANGLE/accuracyFactor;
	int a, b, c, d;
	float af, bf, cf, df;
	bool endOfFile = 0;

	unsigned char CSpace[37][37][37][37];

	//init C-Space with 1's
	for(a=0; a<x; a++) {
		for(b=0; b<x; b++) {
			for(c=0; c<x; c++) {
				for(d=0; d<x; d++) {
					CSpace[a][b][c][d] = 255;
				}
			}
		}
	}

	//open file
	myFile = "./test1/data1";
	sourceFile.open(myFile);
	
	//loop for length of file
	while(!endOfFile) {
		//read 5 lines from the file
		
		//line 1 == a
		getline(sourceFile, line);
		af = stof(line, NULL);
		//cout << "af = " << af << endl; //FOR TESTING

		//line 2 == b
		getline(sourceFile, line);
		bf = stof(line, NULL);
		//cout << "bf = " << bf << endl; //FOR TESTING

		//line 3 == c
		getline(sourceFile, line);
		cf = stof(line, NULL);
		//cout << "cf = " << cf << endl; //FOR TESTING

		//line 4 == d
		getline(sourceFile, line);
		df = stof(line, NULL);
		//cout << "df = " << df << endl; //FOR TESTING

		//line 5 == ; divider between coordinates
		getline(sourceFile, line);
		//cout << ";" << endl;

		//convert negative degrees to corresponding positive angle
		if(af < 0) {
			af = 360 + af;
		}

		if(bf < 0) {
			bf = 360 + bf;
		}

		if(cf < 0) {
			cf = 360 + cf;
		}

		if(df < 0) {
			df = 360 + df;
		}

		//convert to ints
		a = floor(af/10);
		b = floor(bf/10);
		c = floor(cf/10);
		d = floor(df/10);

		//check floored values against original FOR TESTING
		cout << a << "/" << af << " - " << b << "/" << bf << " - " << c << "/" << cf << " - " << d << "/" << df << endl;

		//mark array[a][b][c][d] = 0
		CSpace[a][b][c][d] = 0;

		//check if eof has been reached
		if(sourceFile.eof()) {
			cout << "Reached eof - check 3" << endl;
			endOfFile = 1;
		}

		//check if eof has been reached
		if(sourceFile.peek() == 10) {
			cout << "Reached eof - check 2" << endl;
			endOfFile = 1;
		}
	}

	//close source file
	if(sourceFile.is_open()) {
		sourceFile.close();
	}

	/*
	//print array
	for(a=0; a<1; a++) {
		for(b=0; b<37; b++) {
			for(c=0; c<37; c++) {
				for(d=0; d<37; d++) {
					cout << CSpace[a][b][c][d];
				}
				cout << endl;
			}
			cout << endl;
			cout << endl;
		}
	}
	*/

	//store array in file
	outfile.open("cspace");
	for(a=0; a<x; a++) {
		for(b=0; b<x; b++) {
			for(c=0; c<x; c++) {
				for(d=0; d<x; d++) {
					outfile << (int)CSpace[a][b][c][d];
				}
			}
		}
	}

	/*
	//store 0 locations in file for plotting
	outfile.open("cspacePlotData");
	for(a=0; a<x; a++) {
		for(b=0; b<x; b++) {
			for(c=0; c<x; c++) {
				for(d=0; d<x; d++) {
					if(CSpace[a][b][c][d] == 0){
						outfile << b << " " << c << " " << d 
							<< endl;

					}
				}
			}
		}
		outfile << "\n";
	}
	*/
	
}

\end{lstlisting}
\subsubsection{convert.cpp}
\begin{lstlisting}[language=C++]
#include <stdio.h>
#include <stdlib.h>
#include <iostream>
#include <fstream>
#include <string.h>
#include <cstring>
#include <time.h>
#include <unistd.h>
#include <math.h>

using namespace std;

/************************************************************************************************
 * Program: Target point converter
 * Author: Helena Bales (Inverse Kinematics by Brett Moffatt)
 * Date: May 17th, 2017
 * Description: For the RockSat-X Hephaestus payload. Converts our target points from cartesian 
 * 	form into a representation within the configuration space. This is accomplished through 
 * 	Inverse Kinematics with theta0 isolated to four discrete values, allowing us to resolve 
 * 	the results of IK from a 4D area in the configuration space to four possible solutions,
 * 	of which at least one must be marked as 0 in the cspace.
 * Input: ./targets - A text document containing one target (in the form: x y z) per line, in 
 * 		cartesian form
 * Output: ./thetaTargets - A text document containing one target per line (in the form: 
 * 	theta0 theta1 theta2 theta3), where theta0-3 refers to the rotational angle from the 
 * 	defined normal
 ***********************************************************************************************/

/************************************
 * theta1 = roll = {0, 90, 180, 270}
 * theta2 = yaw = tan-1(yo/xo)
 * theta3 = pitch1
 * theta4 = pitch2
 ***********************************/

int main() {
	char* myFile;
	ifstream sourceFile;
	ofstream outfile;
	char line[24];
	char* tokens;

	int xo, yo, zo;	//real target point
	int theta0, theta1, theta2, theta3; //cspace target point
	double l1, l2; //arm segment lengths
	double k1, k2, gamma; //variables for calculating pitch
	double t, b, fraction1, pt1, pt2; //some more variables for calculations
	bool endOfFile = 0; //marker for eof

	//define arm lengths
	l1 = 3.99;
	l2 = 4.49;

	//open the file of targets
	myFile = "./targets";
	sourceFile.open(myFile);

	//open the file for theta targets
	myFile = "./thetaTargets";
	outfile.open(myFile);


	while(!endOfFile) {
		//read a line
		sourceFile.getline(line, 24);

		//break the line on spaces
		tokens = strtok(line, " ");

		//store the values as integers
		xo = stoi(tokens);
		tokens = strtok(NULL, " ");

		yo = stoi(tokens);
		tokens = strtok(NULL, " ");

		zo = stoi(tokens);
		tokens = strtok(NULL, " ");

		//check if it is eof
		if(sourceFile.eof() || sourceFile.peek() == 10) {
			endOfFile = 1;
			cout << "end of file reached" << endl;
		}

		//calculate variables
		t = (xo*xo) + (yo*yo) - (l1*l1) - (l2*l2);
		b = 2 * l1 * l2;
		fraction1 = t/b;
		pt1 = sqrt(1 - (fraction1 * fraction1));
		pt2 = fraction1;

		//calcualte theta2
		theta2 = atan2(zo, sqrt((yo*yo) + (xo*xo)));

		//calculate k1, k2, and gamma
		k1 = l1 + (l2 * cos(theta2));
		k2 = l2 * sin(theta2);
		gamma = atan2(k1, k2);

		//calculate theta0, theta1, theta3
		theta1 = atan(yo/xo);
		theta3 = atan2(yo, xo) - gamma;
		theta0 = 0; //TODO move this motor to change the plane

		//print the thetas to the file
		outfile << theta0 << " " << theta1 << " " << theta2 << " " << theta3 << " " << endl;

	}

	//close the file
	if(sourceFile.is_open()) {
		sourceFile.close();
	}

	//close the file
	if(outfile.is_open()) {
		outfile.close();
	}


	return 0;
}
\end{lstlisting}
\subsubsection{pathing.cpp}
\begin{lstlisting}[language=C++]
\end{lstlisting}
\subsection{Data Storage}
\subsubsection{SDRead.py}
\begin{lstlisting}[language=Python]
import matplotlib.pyplot as plt
import sys

if len(sys.argv) != 2:			# Make sure filename is specified on command line
	print("No file specified.")	
	print("USAGE:", sys.argv[0], "<filename>")
	sys.exit(1)

temps = []							# Create an empty list of temperatures

with open(sys.argv[1], "br") as f:	# Open output file for binary reading
	data = bytes(f.read(3))			# Read three bytes
	while data != bytes(b""):		# Make sure we have input (i.e. not EOF)
		num = (data[0] * pow(2, 8))	# Load the most significant byte
		num += (data[1])			# Load the least significant byte
		temps.append(num)			# Store the number in the array
		data = bytes(f.read(3))		# Get thee next three bytes
	
scaled = [(i/(2**10)) for i in temps] # Scale the temperature values to between 0-1
plt.plot(scaled)					# Create the plot
plt.ylabel("Relative temperature")	# Always label your axises
plt.xlabel("Sample number")			
plt.show()							# Display the plot
\end{lstlisting}
\subsubsection{telemetry.c}
\begin{lstlisting}[language=C]
/*
 * telemetry.c
 *
 * Created: 1/18/2017 10:44:49 AM
 *  Author: Michael Humphrey
 */ 

#include "telemetry.h"
#include "RSXAVRD.h"
#include <avr/eeprom.h>
#include <string.h>

uint16_t current_address;

void telemetry_init(void) {
	current_address = eeprom_read_word(0);
	if (current_address == 0xFFFF) {
		current_address = 2;
	}
}

// Sends a code with the predefined TELEMETRY_TIME constant.
void inline telemetry_send_code(uint8_t code) {
	send_code(code, TELEMTRY_TIME);
}

// Log a message to the EEPROM
void eeprom_log(char* message) {
	eeprom_update_block(message, (void *) (uint16_t) current_address, strlen(message)+1);
	current_address += strlen(message)+1;
	eeprom_update_word(0, current_address);
}


// Log a message to the SD card - DEPRECATED
/*
void SD_log(char* message) {
	// Write elapsed time to SD card
	char *result = (char*) malloc(6*sizeof(char)); // Max value of get_time() is 65535, which is 5 characters, plus 1 for null terminator

	if (result == NULL)
		return;

	itoa(get_time(), result, 10);
	// Write result string to SD card
	free(result);

	// Write separator to SD card

	// Write message to SD card

	// Write line terminator to SD card
}
*/
\end{lstlisting}
\subsection{Main}
\subsubsection{main.c}
\begin{lstlisting}[language=C]
/*
 * Hephaestus.c
 *
 * Created: 1/17/2017 6:05:07 PM
 * Author : Michael Humphrey
 */ 

#include <avr/io.h>
#include "phases.h"
#include "RSXAVRD.h"
#include "retract.h"


int main(void)
{
	// Pin configuration
	AVR_init();

	uint8_t status = 0;

	//timer_counter_enable(1,1); // code that enables timer event - Jon says Michael wanted this?

	timer_event_enable(0,1); // enables timer event line 0

	timer_event_enable(1,1); // enables timer event line 1

    /* Replace with your application code */
    while (1) 
    {
		// Phase 1: Idle
		status = idle();

		// Phase 2: Observation
		status = observation();

		// Phase 3: Science
		status = science();
		if (status != 0) { // if our arm is not calibrated i.e. collapsed and in home position...
			retract(); // turn off all motors and retract into a safe position
		}

		// Phase 4: Off
		status = off();
		// Under normal circumstances, off never returns.
		// Off only returns if there was an error.

		// Phase 5: Safety
		status = safety();
    }
}
\end{lstlisting}
\subsubsection{phases.h}
\begin{lstlisting}[language=C]
/*
 * phases.h
 *
 * Created: 1/20/2017 3:36:58 PM
 *  Author: Michael Humphrey
 */ 


#ifndef PHASES_H_
#define PHASES_H_

int idle(void);			// Phase 1
int observation(void);	// Phase 2
int science(void);		// Phase 3
int off(void);			// Phase 4
int safety(void);		// Phase 5
void retract(void);  // retract phase

#endif /* PHASES_H_ */
\end{lstlisting}
\subsubsection{Modes of Operation}
\subsubsubsection{idle.c}
\begin{lstlisting}[language=C]
/*
 * idle.c
 *
 * Created: 1/20/2017 3:36:32 PM
 *  Author: Michael Humphrey
 */

#include "RSXAVRD.h"
#include <avr/io.h>
#include <avr/interrupt.h>
#include "telemetry.h"

uint8_t ready = 0;

ISR(INT6_vect) {
	ready = 1;
}

 // First phase of payload deployment. Wait for TE-R lines to go high, then return.
 int idle(void) {
	
	// Infinite loop until receive signal to start


	while (!ready) {}

	eeprom_log("idle phase complete");

	return 0;
}
\end{lstlisting}
\subsubsubsection{observation.c}
\begin{lstlisting}[language=C]

/*  observation.c
 *  Created: 1/22/2017 5:12:25 PM
 *  Author: Amber Horvath
 */

#include <avr/io.h>
#include "phases.h"
#include "RSXAVRD.h"
#include "telemetry.h"
#include "MOTOR_DEF.h"

int observation(){

     
      camera_enable(POWER_ON); // turns on camera

      motor_pwr(MOTOR_CAMERA, POWER_ON); // power on the motor for the camera

      motor_dir(MOTOR_CAMERA, CLOCKWISE); // set the camera to move clockwise

      motor_step(MOTOR_CAMERA, DEGREES_TO_STEPS(180), 1, 85);

      _delay_ms(500);

      motor_dir(MOTOR_CAMERA, COUNTER_CLOCKWISE);

      motor_step(MOTOR_CAMERA, DEGREES_TO_STEPS(360), 1, 85);

      _delay_ms(500);

      motor_dir(MOTOR_CAMERA, CLOCKWISE);

      motor_step(MOTOR_CAMERA, DEGREES_TO_STEPS(180), 1, 85);

      _delay_ms(500); 

      telemetry_send_code(OBSERVATION_PHASE); // let us know we finished Observation mode

      eeprom_log("observation phase complete");

      return 0;


}

\end{lstlisting}
\subsubsubsection{science.c}
\begin{lstlisting}[language=C]
#include <avr/io.h>
#include "phases.h"
#include "RSXAVRD.h"
#include "telemetry.h"
#include <avr/interrupt.h>
#include <util/delay.h>
/*
* Design of science phase: arm will move to touch sensor and press it. Upon moving the arm to the location, we will
* check whether the touch sensor was pressed and change our status to represent whether or not it was pressed.
* We will also write a code over the telemetry lines to represent whether or not the touch sensor was pressed.
*/

int science(){
	int status = 0;
	int M1_POS, M2_POS, M3_POS, M4_POS;
	int M1_NXT, M2_NXT, M3_NXT, M4_NXT;
	int M1_DIF, M2_DIF, M3_DIF, M4_DIF;
	int M1_STP, M2_STP, M3_STP, M4_STP;
	int scale_factor = 0.225;
	char path_step;

	//TODO reference CSpace in program memory (the address of a 4D array in program memory)
	char**** cspace;
		
	// init motor values to 0
	M1_POS = 0;
	M2_POS = 0;
	M3_POS = 0;
	M4_POS = 0;

	// turn on camera
	camera_enable(1);

	// power on motors
	motor_pwr(1, 1);
	motor_pwr(2, 1);
	motor_pwr(3, 1);
	motor_pwr(4, 1);

	//repeat for length of path
	while(1) {
		//set motor direction to be forward
		motor_dir(1, 0);
		motor_dir(2, 0);
		motor_dir(3, 0);
		motor_dir(4, 0);
		
		//find the next point in the path
		path_step = cspace[M1_POS][M2_POS][M3_POS][M4_POS];

		//set the lower and upper bounds for the loops so only indeces 
		//in range are checked
		if(M1_POS == 0) { sw = 0; } else { sw = -1; }
		if(M2_POS == 0) { sx = 0; } else { sx = -1; }
		if(M3_POS == 0) { sy = 0; } else { sy = -1; }
		if(M4_POS == 0) { sz = 0; } else { sz = -1; }

		if(M1_POS == 36) { ew = 1; } else { ew = 2; }
		if(M2_POS == 36) { ex = 1; } else { ex = 2; }
		if(M3_POS == 36) { ey = 1; } else { ey = 2; }
		if(M4_POS == 36) { ez = 1; } else { ez = 2; }


		//check each of the neighboring points
		for(int w=sw; w<ew; w++) {
			for(int x=sx; x<ex; x++) {
				for(int y=sy; y<ey; y++) {
					for(int z=sz; z<ez; z++) {
						if(cspace[M1_POS+w][M2_POS+x][M3_POS+y][M4_POS+z] == path_step + 1) {
							M1_NXT = M1_POS + w;
							M2_NXT = M2_POS + x;
							M3_NXT = M3_POS + y;
							M4_NXT = M4_POS + z;
							break;
						}
					}
				}
			}
		}

		if(M1_NXT == M1_POS && M2_NXT == M2_POS && M3_NXT == M3_POS && M4_NXT == M4_POS) {
			//end of path has been reached
			//TODO send eop telemetry signal
			return 0;
		}

		//change _NXT values to be less than 360
		if(M1_NXT > 359) {
			M1_NXT = M1_NXT - 359;
		}

		if(M2_NXT > 359) {
			M2_NXT = M2_NXT - 359;
		}

		if(M3_NXT > 359) {
			M3_NXT = M3_NXT - 359;
		}

		if(M4_NXT > 359) {
			M4_NXT = M4_NXT - 359;
		}
		
		// calculate change in each motor (in degrees)
		M1_DIF = M1_NXT - M1_POS;
		M2_DIF = M2_NXT - M2_POS;
		M3_DIF = M3_NXT - M3_POS;
		M4_DIF = M4_NXT - M4_POS;

		// reverse motor direction for negative difference
		if(M1_DIF < 0) {
			motor_dir(1, 1);
			M1_DIF = M1_DIF * -1;
		}

		if(M2_DIF < 0) {
			motor_dir(2, 1);
			M2_DIF = M2_DIF * -1;
		}

		if(M3_DIF < 0) {
			motor_dir(3, 1);
			M3_DIF = M3_DIF * -1;
		}

		if(M4_DIF < 0) {
			motor_dir(4, 1);
			M4_DIF = M4_DIF * -1;
		}

		// convert degress to steps
		M1_STP = M1_DIF / scale_factor;
		M2_STP = M2_DIF / scale_factor;
		M3_STP = M3_DIF / scale_factor;
		M4_STP = M4_DIF / scale_factor;

		// move each motor that many steps
		motor_step(1, M1_STP, 28, 99);
		motor_step(2, M2_STP, 28, 99);
		motor_step(3, M3_STP, 28, 99);
		motor_step(4, M4_STP, 28, 99);

		// set the current position
		M1_POS = M1_NXT;
		M2_POS = M2_NXT;
		M3_POS = M3_NXT;
		M4_POS = M4_NXT;

		if(touch_sensor_check() == 0x01){
			telemetry_send_code(TOUCH_SENSOR_1_ENGAGED); // For now, always send TOUCH_SENSOR_1_ENGAGED.
		}

		// code to return arm to callibrated "home" position
		if(get_calibration_status() != 0x01){ // or whichever motor refers to the home position
			status = 1;
		}

	}

	return status;

}
\end{lstlisting}
\subsubsubsection{retract.c}
\begin{lstlisting}[language=C]
#include <avr/io.h>
#include <avr/interrupt.h>
#include <util/delay.h>
#include "RSXAVRD.h"
#include "phases.h"
#include "retract.h"
#include "MOTOR_DEF.h"
#include "telemetry.h"

uint8_t plate_retracted_flg; // flag to keep track of plate's position

ISR(INT5_vect){
	
	if(plate_retracted_flg == 0x00){
		retract();
	}

}

void retract(){ 

	motor_pwr(MOTOR_DECK_PLATE, POWER_ON);
	
	_delay_ms(500); // delay for motor after powering on

	motor_pwr(MOTOR_CAMERA, POWER_OFF); // turn off all other motors
	motor_pwr(MOTOR_DECK_ARM, POWER_OFF);
	motor_pwr(MOTOR_PAN, POWER_OFF);
	motor_pwr(MOTOR_SHOULD, POWER_OFF);
	motor_pwr(MOTOR_ELB, POWER_OFF);

	camera_enable(POWER_OFF); 
	
	motor_dir(MOTOR_DECK_PLATE, COUNTER_CLOCKWISE); // rotates the deck plate to 

	motor_step(MOTOR_DECK_PLATE, 1650, 28, SPEED + 19); // the amount of steps needed to pull the arm back in

	eeprom_log("deck plate has been retracted");

	plate_retracted_flg = 0x01;


}

void extend(){

	motor_pwr(MOTOR_DECK_PLATE, POWER_ON); // powers on deck plate motor

	_delay_ms(500); // delays to allow motor to power on

	motor_dir(MOTOR_DECK_PLATE, CLOCKWISE); // push the deck plate out

	motor_step(MOTOR_DECK_PLATE, 1650, 28, SPEED + 19); // the amount of steps needed to move the deck plate at a good speed

	plate_retracted_flg = 0x00; // plate is NOT retracted


}
\end{lstlisting}
\subsubsubsection{safety.c}
\begin{lstlisting}[language=C]
/* safety.c
 * 
 * Created by: Amber Horvath
 * Date Created: 1/22/17 5:36:47
 *
 *
 */

#include <avr/io.h>
#include "phases.h"
#include "retract.h"
#include "telemetry.h"

int safety(void){
    
	eeprom_log("in safety - attempting retract");

    retract();

    while(1){}; // abort mission, we are hanging here

}
\end{lstlisting}
\subsubsubsection{off.c}
\begin{lstlisting}[language=C]
#include "telemetry.h"

int off(void) {

	eeprom_log("power off");

	while(1){};

	// This return statement should never be reaches, so return error.
	return 1;
}
\end{lstlisting}
