\documentclass[letterpaper,10pt]{article}

\usepackage{geometry}
\usepackage{hyperref}
\usepackage[nopostdot]{glossaries}
\usepackage[pdftex]{graphicx}
\usepackage{tikz}
\usepackage{wrapfig}
\geometry{textheight=8.5in, textwidth=6in}
\newenvironment{bottompar}{\par\vspace*{\fill}}{\clearpage}

\makeglossaries
\loadglsentries[main]{Glossary}

\title{Amber Weekly Progress Report}
\author{Amber~Horvath}

\parindent = 0.0 in
\parskip = 0.1 in

\begin{document}
\maketitle

\subsubsection{Week 3}
This previous week, our team made great strides by writing up our problem statement, which served useful in nailing
 down what we hope to accomplish this coming term, and how we can actualize these goals. We also took a group photo 
 with our full team (comprised of us three computer scientists, six mechanical engineers, and three electrical 
 engineers). This photo was required by Dr. Nancy Squires, our advisor. We also presented our plans in the form of a 
 powerpoint to our Colorado collaborators over a group call.

For the next week, we are going to meet with our team to discuss future prospects for our project. Since we've chosen 
a design, our next step is to begin prototyping implementation designs and begin research on the telemetry aspect of 
the project. We are also going to use our RockSat-X project as a stepping stone to introducing a designated lab for 
satellite creation to Oregon State University. In order to do that, however, we will need to work on finding space on
 campus for such a lab and lobbying for its creation. While this might seem outside of the scope of the project, it 
 would be greatly beneficial for future students who have an interest in space exploration and development.

So far, we have not encountered any barriers. The only issue is coordinating schedules and planning with 11 other 
students. Finding times where everyone is available is difficult, but since we have an effective group chat and a
 Google Calendar with everyone's individual schedules filled in, we've been doing our best to mitigate this issue.

\subsubsection{Week 4}
This week, our team worked to figure out more logistical details. We assembled two new sub-teams, one is the budget 
planning committee (to which I represent the CS group), and a group to prepare a slideshow presentation for an event 
happening next week. As for progress on the tool, we just recruited a new post-grad to help who has a degree in 
mechanical engineering but is coming back to school for a computer science degree. He is currently enrolled in CS 161 
and thus has experience in C++ and nothing else. 

Our next initiative is to meet up with the electrical engineers to figure out constraints between hardware and 
software and read up on some of the documentation on what we can load into our payload as far as the robotics side 
goes.

Our barriers have been limited to task and time managing. Between the amount of students, the amount of deadlines, and 
the differing levels of expertise in each others domains, it can be hard to keep communication clear and comprehensive.
Meeting times have also been difficult to coordinate due to the different schedules.

\subsubsection{Week 5}
This week we refined our problem statement and created a rough draft of the Requirements document. I was unavailable 
to attend our weekly meeting, but I did attend the first budget committee meeting. The meeting was held over the app 
Discord, and we came up with a budget for the rest of our project's lifespan.

Our next initiatve is to gather information from the electronics team to gather information for our requirements
document. This is important, as hardware limitations will greatly impact our software design choices.

The problems we have encountered include illness and traveling. When we do not have teammates available, it can be
difficult to be as productive due to reduced man power. Hopefully everyone comes back happy and healthy soon.

\subsubsection{Week 6}
This week, our team worked on finalizing our Requirements Document, along with doing our first Product Design Review 
meeting with our collaborators in Colorado. We also moved forward on some key design choices, such as choosing to 
program in C on an ATMega128 microcontroller. We also worked on communication between our teams by bringing in Discord 
(the app) as a new medium of communication.

Next week, we hope to work on the framework for the software. With a key design choice in place, we can make progress
towards actual implementation. This is an exciting turning point in our project!

Problems included communication outside of our sub-team. It can be hard to communicate within the larger group.

\subsubsection{Week 7}
This week, our team worked on finalizing our Tech Review and continued collaboration with the larger RockSat-X team. 
Together, we're working to reach conclusions on how the design will be so we can hopefully dive into implementation 
within the next couple weeks. We've settled upon most choices, but it is difficult to reach conclusions with the rest 
of the team about where we stand and how everything will fit together. We did divide work between the three of us
for the software components that are necessary to the overall design of the project. 

Next week I will be out of town for most of the week, so I am leaving most decision making to the rest of my team to 
figure out as far as moving forwards.

Our main issues include traveling as that complicates our established workflow and coordinating all of our deadlines,
both for this class and other classes.

\subsubsection{Week 8}
This week, we finalized our Tech Review. Most of the week I was out of town, but we are setting up our relations with 
the electrical team for next term and figuring out meeting times for our larger group next term.

Next week we hope to meet with the electrical team, but this may be difficult to accomplish as it is Thanksgiving
and a lot of people will be heading out of town to meet with family.

Issues encountered included figuring out specifics regarding the implementation efforts that will need to be made.
Looking into potential solutions and having to deal with some of the ambiguities about the hardware solutions
presented so far made figuring out the software solutions difficult. However, as more information is
presented about hardware solutions, this issue should resolve itself.

\subsubsection{Week 9}
This week was a bit of an odd week due to the Thanksgiving holiday. However, despite the holidays, our team still met 
up to discuss planning how we will finish the term. There still lies some action items that must be completed such as 
resolving issues on our Tech Review document, preparing for the Critical Design Review, and doing our end of term 
project assigned for this class. We will be meeting on Tuesday of this next week to further our plans and see how 
much we have accomplished over the break.

Next week will include meeting to discuss where we stand as a group and what work needs to be completed by what date.
As the end of the term approaches, everyone is more busy and juggling many assignment deadlines from other classes
so touching bases to ensure that the work required for this class gets done.

The main issue this week was scheduling conflicts due to Thanksgiving and the aforementioned deadlines drawing
close. Staying on top of work is sometimes harder than one would imagine.

\subsubsection{Week 10}
This week our Project Design Document was due. Due to the business of Thanksgiving and other classes interfering, 
it was rather difficult getting this document done and over with. However, we managed to pull through. We also
made progress on getting ready for next term with the larger senior design group. We came up with a meeting time
for next term and made plans to meet up with the electrical engineers once more to work out specifics
of our design choices.

Next week we will need to submit our progress report (which will include entries such as this one) and do
our Critical Design presentation for the people in Colorado. Our team will most likely meet to work on the
project over winter break, but specific times for that have not been chosen yet.

The main issues this week included miscommunications regarding work balance and meeting times. Some
members were unable to attend previously scheduled meetings, resulting in work not being distributed
properly and a lot of questions were fielded online as opposed to in person, which slowed our workflow.
However, all work was eventually submitted on time, but it was a more difficult road to complete this then
what we hope to achieve in the future.