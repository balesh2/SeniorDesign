\subsubsection{Week 3}
\begin{itemize}
\item{
\textbf{Progress}

This week I made significant strides in the design of our project. I wrote part of the Project Definition assignment. I started the Project Description with a description of the problem, broken down into the requirements of the RockSat-X program and the payload that we decided on for the project. In order for our senior design project to be successful, we have to build the payload, meet the RockSat-X project requirements (such as testing, documentation, and design reviews), and meet the capstone class requirements. Our payload idea is a mechanical arm, and as a project it is capable of meeting all the requirements.

While the Project Definition document met our capstone class requirements for the week, there were also RockSat-X requirements to be met this week. The RockSat-X CoDR (Conceptual Design Review) was this week. As a large group (including two teams of ME's, one team of EE's, and the CS team) we developed the CoDR powerpoint that was presented yesterday to RockSat-X. This document included all of our conceptual payload designs thus far, and was our first time presenting our designs to the RockSat-X group. Following that presentation, in order to meet the RockSat-X requirements, we took a group photo.

In addition to the RockSat-X requirements and the capstone class requirements, we met the payload requirements by meeting with Nancy Squires to discuss the project, get approval of the Project Definition assignment, and discuss starting an official Space Lab at OSU. The CoDR presentation is available here: https://github.com/balesh2/SeniorDesign/blob/master/presentations/CoDR-Hephaestus.pdf
}

\item{
\textbf{Plans}

The next week will be focusing on the development of documents for Senior Design class as well as for the RockSat-X project. Specifically, we will be revising the Project Description document and begin the Requirements Document. We will also be continuing the design process for the payload with the other teams.
}

\item{
\textbf{Problems}

None.
}
\end{itemize}

\subsubsection{Week 4}
\begin{itemize}
\item{
\textbf{Progress}

This week I was at the Grace Hopper Celebration of Women in Computing. I did not do any work directly on the RockSat-X project, but I did talk to many people about Computer Science and space exploration.
}
\item{
\textbf{Plans}

Next week will be focusing on the development of the requirements document for Senior Design and PDR presentation. The PDR presentation is coming up and will require us to compile a powerpoint about our design, practice presenting it, and presenting it for the RockSat-X program.
}
\item{
\textbf{Problems}

I encountered a significant obstacle to completing work this week. I did not have internet access at Grace Hopper, so I was unable to work on the project or create an update.
}
\end{itemize}

\subsubsection{Week 5}
\begin{itemize}
\item{
\textbf{Progress}

This week was focused on developing the requirements document for Hephaestus and revising the Project Description document. The revision of the Project Description document was turned in on Wednesday after adding more of a focus on the software side of the project. The first draft of the requirements document will be turned in by the end of the day today. I focused on creating the outline of the document and writing the introduction. The introduction establishes a purpose and description of the document, an overview of the mission description, the mission success criteria, and the priorities for the requirements. The rest of the document describes the functional and non functional requirements that we have established for the software that controls the Hephaestus payload.
}
\item{
\textbf{Plans}

The next week will focus on creating a solid final draft of the Requirements Document and presenting PDR. That will require meeting as a group to practice presenting PDR and meeting as a group to present PDR.
}
\item{
\textbf{Problems}

Availability has been a problem this week. It has been a challenge to fit all of the large group meetings into my schedule and still have time to catch up on homework after Grace Hopper and work.
}
\end{itemize}

\subsubsection{Week 6}
\begin{itemize}
\item{
\textbf{Progress}

This week, work focused on the development of the Requirements Document for Senior Design and finalizing the PDR presentation for RockSat-X. I mainly focused on the Requirements Document, and did significant work on the structure and content of that document. We turned in a draft first, then flushed it out to a final document that was turned in on Friday of Week 6. I focused on the functional requirements, introduction, and structure of the paper. For the PDR presentation, we had to develop requirements and a plan to meet the requirements. There was a lot of overlap in content between PDR and the Requirements Documents, which was ideal for finishing both of these big documents in the same week. In preparation for this presentation, we had one meeting where we all went over content and one where we practiced the presentation. The final presentation for PDR (Preliminary Design Review) was at 7am on Thursday of week 6. Finally, I revised the README for this repository, so that it was more informative regarding the structure, contents, and context of this repository.
}
\item{
\textbf{Plans}

Next week will focus on finalizing major design choices and developing the technical review. The design choices that need to be finalized include the method for assigning test points and the operational modes of the arm.
}
\item{
\textbf{Problems}

None.
}
\end{itemize}

\subsubsection{Week 7}
\begin{itemize}
\item{
\textbf{Progress}

This week we are developing the Technical Review Document for Senior Design. As such, we have divided the requirements up between the three of us as follows:

Amber Horvath:

    Emergency Expelling of Payload

    Program Modes of Operation

    Target Success Sensors

Helena Bales:

    Target Generation

    Movement of arm

    Arm position tracking

Michael Humphrey:

    Telemetry

    Video Camera

    Data visualization and processing

Each of us shall be responsible for insuring the completion of their assigned tasks. We will focus on our assigned tasks for the tech review. This week has focused on defining and assigning the requirements to each of us. We have also finalized some design choices, specifically in the modes of operations, emergency procedures, and arm target generation.
}
\item{
\textbf{Plans}
}
\item{
\textbf{Problems}
}
\end{itemize}

\subsubsection{Week }
\begin{itemize}
\item{
\textbf{Progress}
}
\item{
\textbf{Plans}
}
\item{
\textbf{Problems}
}
\end{itemize}

\subsubsection{Week }
\begin{itemize}
\item{
\textbf{Progress}
}
\item{
\textbf{Plans}
}
\item{
\textbf{Problems}
}
\end{itemize}

\subsubsection{Week }
\begin{itemize}
\item{
\textbf{Progress}
}
\item{
\textbf{Plans}
}
\item{
\textbf{Problems}
}
\end{itemize}
