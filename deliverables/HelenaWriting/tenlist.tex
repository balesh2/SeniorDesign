\documentclass[letterpaper,10pt]{article}

\usepackage{geometry}
\usepackage{hyperref}
\usepackage[nopostdot]{glossaries}
\usepackage[pdftex]{graphicx}
\usepackage{tikz}
\usepackage{wrapfig}
\geometry{textheight=8.5in, textwidth=6in}

\title{Ten Pieces of Advice for Starting Capstone\\ In a Cross-Disciplinary Group}
\author{Helena Bales \\ CS463 - Spring 2017}

\parindent = 0.0 in
\parskip = 0.1 in

\begin{document}
\maketitle

\begin{enumerate}
\item{
\textbf{Meet with the whole team every week} - Meeting with the whole team, including the teams from other majors, every week will help to keep everyone on the same page. When many people are coming at the project from different perspectives, it is important to give yourself the time to communicate effectively in person. Meeting every week will also help with team morale.
}
\item{
\textbf{Start fundraising immediately} - Hardware projects require some money generally, so create a budget and raise the money as early as possible. You do not want to be worried about money late in the project.
}
\item{
\textbf{Start several ongoing group messages} - Communicating with many people can be a challenge, so use a tool like GroupMe or Slack for group messages. Have at least one group with everyone and one group for each Capstone group and major (for example CS, EE, ME1, ME2, and ME). Keep it professional but light, using the occational gif and emoji but leaving out hate speech, condescending language, insults, profanity, etc. The group message is a good place to brainstorm, remind each other of meetings and deadlines, and have jokes. It is an effective way to build a stonger team.
}
\item{
\textbf{Start a group calendar} - You should have two group calendars, one with everyone's schedules and one with the group meetings and work sessions. The calendar of schedules is crutial for getting anything done in a larger group. Scheduling a time where 13 people can meet is challenging, but is impossible without everyone's schedules. A google calendar will make it all easier. The calendar with all the meetings and work times is perfect for that "it's on the calendar" response to missing meetings. It is a good way to hold everyone accountable for attendance without having to remind everyone individually.
}
\item{
\textbf{Start a group google drive folder for all deliverables, documentation, and work notes} - It is essential that everyone have access to all the documentation that exists on the project. Just because some information is about hardware doesn't mean it doesn't affect the software. Dividing the documents by level of formality is also useful. One possible division is deliverables (with things you would turn in, i.e. formal and somewhat to moderately technical), documentation (moderate formality, more technical), and work notes (least formal and most technical). Don't be afraid to use other team's notes during development, because consistency and validation are important to the development process.
}
\item{
\textbf{If your project has a context other than the Capstone class, understand how the timelines relate} -  It can be challenging to reconcile two different project timelines or scopes, and it is possible to have both with projects such as RockSat-X, where the launch is not until August and we started in early September of the previous year. As such, we had already completed some of the initial capstone assignments before the class started without knowing it. This was not a big problem, but it is still a challenge to reconcile the different deadlines. The best way to do so is to have a firm understanding of both sets of requirements.
}
\item{
\textbf{Treat your groupmates with respect and equality} - Each teammate, no matter how different or similar they may appear, has both more in common with you than you think and more individuality than you know. Treating everyone with equality does not mean treating everyone the same, since some individuals have different needs than others (see the next list item), but rather giving everyone the same opportunity to be heard and respected and valued. Being able to recognize the individuality and community within your group will allow you to build a strong group.
}
\item{
\textbf{Empathize with your groupmates and consider their needs} - Understanding the needs of the people that you work with on a daily basis will allow you to be a more effective group. If you know what their needs are, you can offer your help and support where you are able, and will be able to ask for help when you need it without putting them out. With such a long class project, it is important to balance asking for help with shouldering the work alone to maintain a healthy group dynamic and balance of the workload.
}
\item{
\textbf{Find a way to get yourself to class} - You have to do it, so find a way to motivate yourself. Possible motivators include bringing a snack or coffee for yourself and potentially your group (coffee works great for those 8am ones) or to bring something mindless to do, like taking notes and doodling or coloring, so long as it is not distracting to your classmates.
}
\item{
\textbf{Be reliable} - You have to continue working with the same people all year, so get your work done on time and communicate with your team. Show up to class and meetings and respond to emails promptly.
}
\end{enumerate}

\end{document}
