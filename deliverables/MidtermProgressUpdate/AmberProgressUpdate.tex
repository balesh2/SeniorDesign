\subsubsection{Week 1}
This week we attempted to gather ourselves after winter break and make a game plan for this coming term. We met as a 
group and worked on figuring out what needs to get done within the next 5 weeks and set small goals for each coming 
week. We are meeting with our EE collaborators next week to figure out telemetry codes. I've downloaded Atmel studio 
so I can start writing code. We are divvying out work based on our design document from last term.

For the next week, I am hoping to begin work with the telemetry side of the project and help Michael.

So far, we have not encountered any barriers. The only issue is coordinating schedules, which was an issue we 
encountered last term as well. What mitigated that issue was clear and consistent communication across teams, which
our newly established subteams should help with. I am serving as the representative for the Physical Integration team
which is the team in charge of ensuring all the mechanical pieces get manufactured and put together properly. 

\subsubsection{Week 2}
This week was somewhat odd due to the short week the previous week because of snow throwing off our progress and the
 shortness of this week due to Martin Luther King Jr. day. Little progress was made this week. The only progress made
 was attempting to work on our team organizational skills and begin documenting our individual process through a 
 scrum document that Michael set up. 

There was also a change in project delegation as the EE's decided that they want a local storage for the 
microcontroller in the form of an SD card. I will begin looking into lightweight, Atmel-compliant storage systems
that interface with an SD card.

Our barriers have been limited to lack of time between all of us. Also, due to having irregular meetings, we are
having issues knowing what each other are doing and how to keep track of differing obligations and deadlines.
We are attempting to overcome this with the scrum document which should allow for less in-person meetings
and more communication about what we are all doing on our own time. We also are waiting on the EE's for their driver 
code they are writing for us. They have been tasked with writing the C code that will interface with the motors.

\subsubsection{Week 3}
This week we got the code from the EE's for interfacing with the microcontroller. I also did research on SD card 
interfaces with microcontrollers that we are now using to write data from the microcontroller to an SD card. This 
code will be up on Github once we translate it from the Atmega 16 pins to the Atmega 64 (which we are using). I worked
with the EE's to accomplish this translation as they are more familiar with the Atmega 64 archictecture than I am.

Our next step is to make the code compliant with the Atmega64 archictecture. I will also begin work on writing code
that utilizes this library so that we can write telemetry codes, sensor touch data, and other log data to the card
for post-mortem analysis.

The problems we have encountered include illness and traveling. When we do not have teammates available, it can be
difficult to be as productive due to reduced man power. These random holidays also slowed our initial progress, so
it can feel like we are trying to play catch up. We are also nearing deadlines for our STR meeting with the Colorado
RockSat-X group, so I am beginning to feel the pressure of that as I do not feel our team has accomplished as much
as I would like us to.

\subsubsection{Week 4}
This week, I was able to get all the code up on Github for the SD card interface with the microcontrollers. Me and 
the EE's (namely Jonathan Hardman) are working on implementing the remaining functions and pre-processor directives 
for our SD card interface file. We should have that done by next Tuesday. Jonathan also updated the drivers and 
uploaded 
them on Google Drive, so I will import those over to Github. Now that these drivers are almost completely ready, I am 
looking forward to working on more implementation that actually utilizes the libraries and drivers that we have 
gotten. This step, however, hinges on having motors and a working prototype ready so we can determine whether our
arm movements are correct. This will not be ready soon as our funding has been delayed leaving us with no money for 
purchasing these motors.

Next steps include nailing down with Michael what information we are sending over telemetry and what of that 
information should be sent to the SD Card. Next steps also include getting the updated code from Jonathan and getting
 the drivers up on Github and fully implemented.

The CS 462 class is also requiring that we update some of our previous documents and get all our deliverables up onto
One Note, so we will do that as well. Lastly, we have a big meeting with our sponsors in Colorado next week, so we
have some slideshow slides that we need to complete. We are in a stage where we need to start divvying out tasks
and better communicating as a team. We also have dependencies that are reliant upon other teams that we have no way 
of controlling, such as the lack of motors.

\subsubsection{Week 5}
This week I worked on furthering the progress on our SD card implementation. I updated the Makefile such that the 
library now compiles with the rest of the program. Jonathan also updated the code such that it uses the correct pins
and is designed with our Atmega64 in mind. Since the FatFS library is meant to be generalizable across different
microcontrollers, there are some functions that are left as user-designed so that they can be written for specific
microcontrollers. We completed those parts of the code, so the library should be fully functional now.

Our next work will be to test the SD card library. We are also prepping for our STR presentation with our contacts
in Colorado, so we have to complete an extensive PowerPoint skeleton that they provide us. We will also be meeting with
Dr. Smart (a professor in the Mechanical Engineering department) to discuss path-finding for our mechanical arm.

Our main blockers are lack of communication between team members. Despite putting a scrum document into affect and 
requiring each individual team member to add to the document every Monday and Thursday, some members of the team
have been failing to meet this requirement, so it is difficult to know what is going on between other subteams
and what our individual team members are doing on their own time. We also have regularly scheduled meetings, but
attendance has been rather lax on those with team members either forgetting or making excuses right before the meeting
time. Despite us stressing the importance of attending these meetings and communicating relevant information, our team
still struggles with this, so some measures must be taken. However, we are ensure at the time of writing this what
those measures should be. 

\subsubsection{Week 6}
This week was focused primarily on prepping for our big STR presentation. This is a presentation where we review our 
subsystem testing for the RockSat-X group in Colorado. The meeting was at 6 AM on Friday and required us to prepare a 
lot of slides for a slideshow. We also met with Dr. Smart to learn more about preparing our C-Space for 
pathing the arm's movement and determining what are and aren't legal movements for the arm. There are multiple ways
to do this as we can either use inverse kinematics, manually moving the arm through different legal motions and 
recording the different degrees of rotation the motors are in, or using sensors on the motors to determine their 
rotation. We are hoping to do the last option.

Next week we shall work on the presentation for this class. Fortunately, we can lift a lot of the work we did last 
week so it shouldn't be too cumbersome. I will also be meeting with the EE's on Friday to work on testing the SD Card 
implementation and also continue work on the code for moving the arm.

Blockers this week were the fact that I was running a study at IBM so I was out of town most of the weekend and could 
not do work for this since I was busy preparing for that. This should not be a problem now that the study is 
concluded.