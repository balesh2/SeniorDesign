\subsubsection{Week 3}
This past week the Hephaestus project team accomplished several important milestones. We completed our first presentation to the RockSat-X organizers and took a group picture to start raising funding. We are also starting to narrow down our design for the final \gls{payload}.

Because the mechanical and electrical design of the project is not yet finalized, the software team has not yet had any important responsibilities. The electrical team is forbidden from using a device like a Raspberry Pi or an Arduino, so they have decided to use an AVR microcontroller. Amber and I have not used one of these devices, although Helena has. Amber and I will need to start doing research on programming for these devices. We will be using C to program the microcontroller. We won't be able to write any code until the electrical design (i.e. inputs and outputs) are finalized, but we can start creating a software design of how we want the software to work.

No problems have been encountered yet.

\subsubsection{Week 4}
Similar to week 3's blog post, this past week the Hephaestus Software Team did not have any major responsibilities. We attended the Hephaestus team meetings where the mechanical and electrical designs are still being worked out. We are going to have more communication with the Electrical Engineering team to determine the computing platform and computation restrictions. We also began working out budget numbers.

This next week we will be creating several presentations. I will be partly responsible for a 6 minute 40 second presentation to compete for a \$1,000 cash prize. Other fundraising efforts are also in progress. We will also be meeting with the Colorado Space Grant committee for our next presentation for them. We will also need to start working on revising our Problem Statement and start drafting our Requirements document and any other documentation we need.

Currently, the software team is blocked by the electrical team. Until they finalize a design, we cannot start coding. We will be in communication with them, however, to determine what considerations they need to take for the design.

\subsubsection{Week 5}
Since our mechanical and electrical design is still in progress, we have made no progress in the past week toward writing any software. Only work done was finishing the problem statement assignment and drafting our requirements document.

For the next week we will be getting datasheets and other information from the electrical team to aid in drafting our requirements documents. Any limitations of the hardware will be taken into consideration for the software requirements. Those materials should be made available by the electrical team by early next week.

Problems encountered this week were mostly personnel issues. Some of our team has been on vacation and one member is now sick and unable to make it on campus at all. I feel myself coming down with my second illness this term, which will make it even more difficult to get the required signatures we need.

\subsubsection{Week 6}
This week was spent finalizing our software requirements for the project. We did extensive research into the details of the mechanical and electrical design of our \gls{payload} and drew up documents with specifics such as coordinate systems and \gls{payload} layout. We now have a basis for creating our software.

For the next week, I believe we will be able to start writing the framework for the \gls{payload}. We probably won't be able to start programming the actual function of the \gls{payload} until it is built, but we can create the structure of how our software will be laid out.

Some problems were encountered this week with communication outside of our sub-team, but those have been resolved and shouldn't occur again in the future.

\subsubsection{Week 7}
Last week we developed the requirements of our system a lot. We explored technologies that we want to use and confirmed many details with the robotics and electronics team about the requirements of the \gls{payload}. For me, last week was spent primarily going to meetings, relaying information to teammates, and doing research into potential technologies we can use.

Next week, we will hopefully begin implementation of the software. I need to set up a meeting with the electrical team. I can't remember what they want to talk about but we definitely meet as a team with them. Most likely all of the CS team won't be able to make it, and this is a challenge we will need to overcome.

Problems I encountered included finding adequate solutions for the telemetry technology. I thought it would be easy to find several solutions we could use, but it turns out that most of the solutions I found were not compatible with our system for one reason or another. Mostly because none of them actually dealt with the transmission of the data itself, but what it did with the telemetry after it was collected. Other reasons were that they were implemented in the wrong language.

\subsubsection{Week 8}
Last week I helped start our Design Document. We've created the structure for the document and pasted the relevant sections from our previous documents. I set up meetings with the Electrical team and started communication with them to nail down specific software communication requirements. They're going to create a sort of "firmware" for the \gls{payload}, meaning they'll write the code that interacts directly with the hardware, and they'll expose an abstract interface for the Software team so we only need to call something like moveArm(x, y, z); to control the movement of the arm.

Next week we need to finalize the details of how we want to control the \gls{payload} arm. This will probably mean writing an API that we want the Electrical team to implement. We also need to prepare for the CDR coming up in a couple of weeks. This means we need to fill out the slides the Software team is responsible for. There will probably be other work for this presentation that we will tackle as it comes up.

No problems this week.

\subsubsection{Week 9}
This week I didn't do much. The Software team has created an outline for our design document but I haven't added my parts in. I don't foresee it being too much work, as it's mostly already written from the tech review. More details just need to be added. Due to it being Thanksgiving week, I have delayed working on classwork in favor of helping my family prepare for Thanksgiving.

Next week we need to finish our rough draft of the design document as well as write an outline for our presentation.

No problems were encountered this week.

\subsubsection{Week 10}
This week we made a lot of progress finalizing the design for the \gls{payload} software. This was mostly a result of writing the design document. There was much communication with the electrical team.

Next week is finals week. We will be writing our progress report and recording our presentation.

One problem we are encountered is the slow response to questions that arise about the RockSat-X program. I have several questions about the format and delivery of telemetry data that won't be answered until mid- to late-next week. That information was not able to be included in the design document.
