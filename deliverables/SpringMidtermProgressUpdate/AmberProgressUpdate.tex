\subsubsection{Week 1}
\begin{itemize}
\item{
	\textbf{Progress}

	Hi all.

	Sorry, but I had almost no progress this week due to being in Boise due to a family emergency. However, over spring break,
	 I did look into the SD Card issues more and found out that the reason a function was having undefined behavior (returning
	  success and failure seemingly simultaneously) I found that this function was actually being called in a previous library 
	  function call that I had failed to take into account during my stack trace. This was a breakthrough in that it proved 
	  this me and this function weren't losing our "minds" but also left me with the original problem I had of not knowing why 
	  the function was failing in the first place.

}


\item{
	\textbf{Plans}


	The next plan is to continue investigating why this is failing to work and move implementation, hopefully, more
	 towards the arm as the ME's and EE's are getting that in a good place for us to test it.
}


\item{
	\textbf{Problems}


	The main blocker was being out of town and having to focus on my family. Working remotely and while under a lot of 
	emotional pressure is very difficult.
}
\end{itemize}

\subsubsection{Week 2}
\begin{itemize}
	\item{
	\textbf{Progress}


		No notable progress was made week 2 of spring term. I was unfortunately mentally and academically recovering from
		 my absence the week prior and didn't have a lot of time to spend on senior design work. I did meet up with my 
		 team for the first time since the previous term and we designated tasks for the coming weeks including finishing 
		 code implementation for the arm and finishing up the library implementation for the SD card.
	}

	\item{
		\textbf{Plans}


		Upcoming plans include meeting with the ME Brett to figure out where we will place the touch sensors on the arm body and implement code to move the arm to these touch sensor points.
	}

	\item{
		\textbf{Problems}


		Blockers are limited to just the fact that the arm is too heavy to lift and so it will be hard to test. 
		Testing arm movement would require us to manually check the rotation of the motors using a compass. 
		This is not ideal.
	}

\end{itemize}

\subsubsection{Week 3}
\begin{itemize}
\item{
	\textbf{Progress}


	This was a hard week, boys and girls. Mostly due to the fact I was very busy with my research job as we had a paper 
	deadline on Friday and basically had to spend every "free" hour working towards that deadline. This left minimal time 
	to work on senior design, but fortunately I was able to make some notable progress. I met up with Jonathan from the 
	EE's to discuss the issues with the SD card as I was at a loss as to what to do next to determine why that function 
	wasn't working. We reworked the SPI function call but that unfortunately didn't help anything. Our next step is to try 
	reworking our approach and seeing if we can read from a file and perhaps find some commonality between reading and 
	writing where the software breaks to pin-point an exact issue. Our other option is dropping this stretch goal (since 
	that's essentially what it is) and focusing our efforts on writing stable data storage to EEPROM. If we can't get this 
	stuff working soon enough that's probably what we'll have to do. Meanwhile, we also need to start focusing our energy 
	on programming the arm to move. That's a big one.
}

\item{
	\textbf{Plans}


	Next steps are to find out where EXACTLY this code is breaking. And also to implement the retract and move out parts for the deck plate holding the mechanical arm as that's part of my requirements I was assigned ("emergency retraction").

}

\item{
	\textbf{Problems}


	Problems are that this SD card just won't cooperate and I'm so tired of it.
}

\end{itemize}

\subsubsection{Week 4}
\begin{itemize}
	\item{
		\textbf{Progress}


		This week we began working on some more implementation endeavors for the arm's movements. 
		I implemented the touch sensor and deck plate extension and retraction (which were my designated tasks from the 
		Requirements document) as interrupts that execute code upon receiving information from the hardware. 
		
		In the case of the deck plate, an interrupt will be sent upon a timer event line (part of the RSX rocket) going to 
		LOW and signaling the end of our deployment period for the payload. The interrupt will trigger and power the motor 
		attached to the bottom of the deck plate, change it's direction from outward to inwards, and step the motors to 
		pull the plate in. The code for the touch sensors are signaled by the touch sensors being depressed, and will 
		write a code to the telemetry line signaling that we made contact. The implementation for the touch sensor is in 
		science.c while the retract/extend code is in its own file (retract.c and retract.h). Since this is now complete, 
		I will be assisting Helena in her work to finish the arm pathing and movement. Unrelated to the implementation 
		effort, I joined Sam Lundeen (from the MEs) to visit Garmin regarding doing environmental testing for the arm. We 
		met up with Steve Horvath (my dad!) and Greg Fisher to see whether it was viable to mount our arm onto their 
		vibration table and test how the arm faired in situations similar to what will be experienced at launch time. We 
		are going to meet with our team to update them that this is viable and to determine what we want equipped to the 
		arm during this time. While it would be ideal to have everything attached, we've also had funding issues and don't 
		want to risk losing any motors or other valuable pieces that we may not be able to replace if they get damaged 
		during the testing.
	}
	\item{
		\textbf{Plans}


		Future plans are to continue implementation efforts with Helena and I also hope to clean up our repository a bit 
		and add some more documentation regarding how to test our code so when we have the code freeze, the TAs/McGrath 
		aren't confused as to what's going on.
	}
	\item{
		\textbf{Problems}


		Blockers are that I still have been unable to actually test this code due to the motors and arm being not set up 
		currently. Hopefully I will be able to do this by our (extended, thanks McGrath!) May 15th deadline.
	}
\end{itemize}

\subsubsection{Week 5}
\begin{itemize}
	\item{
		\textbf{Progress}


		This week I was extraordinarily busy with other classes so I was unable to do much on the project. I plan to get 
		back to work after my midterm Tuesday of next week.
	}
	\item{
		\textbf{Plans}


		Next week, I plan on getting back into the code and performing minor bug fixes to get the code operational before 
		our code freeze deadline of May 15th. We also have an upcoming meeting with our sponsors in Colorado and expo 
		coming up so things are getting pretty exciting!
	}
	\item{
		\textbf{Problems}


		No stoppers are known as of now.
	}

\end{itemize}
\subsubsection{Week 6}
\begin{itemize}
	\item{
		\textbf{Progress}


		This week, Michael and I worked on fixing some bugs with our implementation of the different phases. We also 
		enhanced the readability of the code by adding some defines within a .h file and refactoring the code to use these 
		more readable names as opposed to the previous iteration which just used numbers like 0 and 1. We believe this 
		makes the code easier for outsiders to understand and, if changes are made in how the implementation works, we can 
		just change the value of the defines as opposed to having to go through and find every number to change. The 
		remaining work is just finishing the science mode, which is Helena's responsibility. We trust her to get this over 
		the finish line.
	}
	\item{
		\textbf{Plans}


		Future progress is to get ready for Expo (May 19th!) and FSMR which should be Wednesday (May 17th) at 11. We also
		 need to finish recording our midterm progress update. Next week will be fun/stressful!
	}
	\item{
		\textbf{Problems}

		
		Blockers are that the arm still isn't moving. We spoke with Huy (a representative from the EE team) who said that 
		we don't know whether the arm will move even in space. This is concerning but ultimately out of our hands.
	}

\end{itemize}
