\documentclass[letterpaper,10pt]{article}

\usepackage{geometry}
\usepackage{hyperref}
\newenvironment{bottompar}{\par\vspace*{\fill}}{\clearpage}
\geometry{textheight=8.5in, textwidth=6in}

\title{Problem Statement For RockSat-X Payload - Hephaestus}
\author{Helena~Bales, Amber~Horvath, and Michael~Humphrey\\ \\ CS461 - Fall 2016}

\parindent = 0.0 in
\parskip = 0.1 in

\begin{document}
\maketitle

\begin{abstract}
The Oregon State University RockSat-X team will demonstrate that an autonomous robotic arm can
locate predetermined targets around the payload under microgravity conditions by using precise
movements. The technical actions performed by this demonstration will illustrate a proof of concept
for creating assemblies, autonomous repairs, and performing experiments in space. In order to
accomplish the Hephaestus mission, the software team shall collect telemetry data and develop the arm
control software. The telemetry shall be sent to the ground station in real time in order to monitor 
the progress of the flight. The software shall be responsible for deploying and moving the arm 
assembly body. Hephaestus will be successful if the arm performs the given motions and if the motions
 are recorded by the on-board video camera and telemetry data.
\end{abstract}

\begin{bottompar}
Approved By
\_\_\_\_\_\_\_\_\_\_\_\_\_\_\_\_\_\_\_\_\_\_\_\_\_\_\_\_\_\_\_\_\_\_\_\_\_\_\_\_\_\_\_\_\_\_\_\_\_\_\_\_\_\_\_\_\_\_\_\_\_\_\_
Date \_\_\_\_\_\_\_\_\_\_\_\_\_\_\_\_\_\_\_\_\_\_\_\_\_\_\_\_
\end{bottompar}

\clearpage

\section{Problem Description}
\subsection{RockSat-X Program Overview}
The Hephaestus Project is part of the RockSat-X 2017 program. This program provides students with a 
rocketry platform on which to launch scientific and technical payloads into Low Earth Orbit (LEO). 
The program is divided into three stages: design, build/test, and integration/flight. This is Oregon 
 State University's first time participating in this program. Participation this year will allow OSU
   to pursue more aerospace projects in the future. The RockSat-X program is based at Wallops Flight 
Facility and funded by the Space Grant Consortium. The rocket will be launched from Wallops and will
 expose students' experiments to space at the rocket's apogee. The program requires that students 
design and build an experimental or technical payload that makes use of microgravity or a space-like
environment.

\subsection{Limitations of Space Travel}
Currently space travel is limited by the need for constructing spacecraft on earth and launching into
orbit. Because of this, the scale of spacecraft is limited to what can be launched by current launch
 vehicles. The limited power of launch vehicles means that traveling beyond LEO has serious mass 
constraints. In order to travel further, we must circumvent this mass limitation. This can be 
solved by launching raw materials and parts and constructing the structure on orbit. 

The International Space Station is currently evaluating possible solutions to this problem. They are 
investigating the efficacy of 3D printing parts on orbit and of performing repairs with the help of 
a large robotic arm. The arm on the ISS has limited functionality because of its size.
 It has the ability to move astronauts around the outside of the ISS but is not capable of the detailed
maneuvers that would be required for it performing repairs itself.

Software limitations are directly related to the limitations of constructing all equipment on earth, prior to launch. Writing code that is robust enough to handle all of the potential situations that can occur in space requires years of planning and foresight. Software that is able to maintain itself is currently not possible, but software that is able to make repairs on hardware autonomously, while difficult, is reasonable.

\section{Proposed Solution}
The Hephaestus payload shall be an assembly containing a mechanical arm capable of performing
intricate maneuvers.
This project is a proof of concept for construction and repair on orbit using a mechanical arm. 
The ability to perform detailed maneuvers using an arm is critical for construction on orbit because
it replaces the current need for a spacewalk to perform the same task, saving time and money and
decreasing risk.

\subsection{RockSat-X Platform for Proof of Concept}
The RockSat-X Platform will be used for this demonstration because of the low barriers to entry and
the future opportunities it will provide OSU for research and study in space.
It is far more cost effective and less error prone to use an already existing rocketry platform for
this demonstration than to attempt to build a rocket dedicated to only the Hephaestus mission.
RockSat-X provides a standardized and consistent platform for building experiments to send into space.
The Hephaestus team only needs to build a payload that meets the design specifications set forth by
RockSat-X to successfully deploy that payload into LEO.
The RockSat-X platform provides 10 cannisters that each contain telemetry and power hookups.
Each cannister may accommodate 1 or 2 experiments.
Additionally, if the Hephaestus mission is a success, it will provide more opportunities for space
research such as CubeSat missions through the Space Grant Consortium.
\subsection{Mechanical Arm For Construction On Orbit}
The Hephaestus project attempts to prove that a mechanical arm is a viable solution for construction on orbit.
This is accomplished by launching a prototype of a small autonomous robotic arm as a RockSat-X payload.
The arm will execute a series of maneuvers to make contact with pre-defined targets on the rocket.
The maneuvers will be observed using the telemetry from sensors and motors on the arm and a video
camera mounted on the payload.
The arm will be controlled using a micro-controller and motors that will allow it to make precise movements.
The deployment and movement of the arm will determine if a mechanical arm is a viable solution to
performing construction and repairs during flight.
\subsection{Software Solution for Arm Control and Telemetry}
The Software for the Hephaestus payload will collect telemetry data, store the video feed from the
camera, and control the arm assembly body motors.
The Software will be responsible for recording every input it receives, output it sends, as well other
important information. The Software will transmit that information to the ground station via the
telemetry hookups.
Additionally, the Software must handle sharing of the telemetry hookups with the neighboring RockSat-X
cannister. 
The Software will also be responsible for controlling the movement of the arm assembly body. To do this, we can load the body with the Atmel software framework which provides drivers and libraries to glue together the pieces of a software design using Atmel devices. Our electrical engineers will be using an Atmega 128 Microcontroller, so our software must be compliant with their device.
Finally, the Hephaestus Software must store the output from the video camera into a persistent storage
location.

\section{Metrics For Success}
\subsection{Mission Success Criteria}
The success of the Hephaestus mission shall be evaluated based on two success
landmarks. These landmarks are the minimum success criteria and the maximum
success criteria. The minimum success criteria defines the minimum task that we
could accomplish and still consider the mission to be successful while the
maximum mission success criteria describes the tasks that would make a fully
successful mission.
\subsubsection{Minimum Mission Success Criteria}
\begin{enumerate}
  \item The arm assembly body shall deploy and a video sweep is successfully
    recorded.
  \item The arm assembly body shall be fully retracted after data collection.
\end{enumerate}
\subsubsection{Maximum Mission Success Criteria}
\begin{enumerate}
  \item The arm assembly body shall deploy and a video sweep is successfully
    recorded.
  \item The arm shall make contact with predetermined targets around the
    payload.
  \item The camera shall record all instances of contact between the arm and the
    targets.
  \item The arm assembly body shall be fully retracted after data collection.
\end{enumerate}
\subsection{Software Success Criteria}
The software can be successful without the mission being successful. For example
in the drastic case where the rocket is destroyed on launch, our software could
still be successful if it met all of the following requirements prior to the
flight.
\begin{enumerate}
  \item The software shall deploy the arm assembly body and perform a video sweep.
  \item Test points shall be generated and stored.
  \item The movements to reach each generated point shall be generated.
  \item The arm shall perform the generated tasks and record the motions using
    the video camera and telemetry channels.
  \item The software shall retract the arm assembly body.
\end{enumerate}
\subsection{Evaluation of Success Criteria}
The mission will be considered successful if the success criteria have been
varifiably met. This means that in order for the mission to be fully successful,
the arm must deploy, perform its motions and retract as well as record all
motions using the video camera and telemetry. We will use the recorded video and
the telemetry data of motor motion and pressure sensors to determine whether the
arm has performed all tasks. Using that data and the benchmarks described in
this document we will determine the level of success of the Hephaestus mission.
\end{document}
