\documentclass[letterpaper,10pt]{article}

\usepackage{geometry}
\usepackage{hyperref}
\geometry{textheight=8.5in, textwidth=6in}

\title{Problem Statement For RockSat-X Payload - Hephaestus}
\author{Helena~Bales, Amber~Horvath, and Michael~Humphrey\\ \\ CS461 - Fall 2016}

\parindent = 0.0 in
\parskip = 0.1 in

\begin{document}
\maketitle

\begin{abstract}
The Oregon State University RockSat-X team will demonstrate that an autonomous robotic arm can
locate predetermined targets around the payload under microgravity conditions by using precise
movements. The technical actions performed by this demonstration will illustrate a proof of concept
for creating assemblies, autonomous repairs, and performing experiments in space. In order to
accomplish the Hephaestus mission, the software team shall collect telemetry data and develop the arm
control software. The telemetry shall be sent to the groundstation in real time in order to monitor 
the progress of the flight. The software shall be responsible for deploying and moving the arm 
assembly body. Hephaestus will be successful if the arm performs the given motions and if the motions
 are recorded by the on-board video camera and telemetry data.
\end{abstract}

\clearpage

\section{}

\end{document}
