\documentclass[letterpaper,10pt]{article}

\usepackage{geometry}
\usepackage{hyperref}
\geometry{textheight=8.5in, textwidth=6in}

\title{Problem Statement For RockSat-X Payload - Hephaestus}
\author{Helena~Bales, Amber~Horvath, and Michael~Humphrey\\ \\ CS461 - Fall 2016}

\parindent = 0.0 in
\parskip = 0.1 in

\begin{document}
\maketitle

\begin{abstract}
The Oregon State University RockSat-X team will demonstrate that an autonomous robotic arm can
locate predetermined targets around the payload under microgravity conditions by using precise
movements. The technical actions performed by this demonstration will illustrate a proof of concept
for creating assemblies, autonomous repairs, and performing experiments in space. In order to
accomplish the Hephaestus mission, the software team shall collect telemetry data and develop the arm
control software. The telemetry shall be sent to the groundstation in real time in order to monitor 
the progress of the flight. The software shall be responsible for deploying and moving the arm 
assembly body. Hephaestus will be successful if the arm performs the given motions and if the motions
 are recorded by the on-board video camera and telemetry data.
\end{abstract}

\clearpage

\section{Problem Description}
\subsection{RockSat-X Program Overview}
The Hephaestus Project is part of the RockSat-X 2017 program. This program provides students with a 
rocketry platform on which to launch scientific and technical payloads into Low Earth Orbit (LEO). 
The program is divided into three stages: design, build/test, and integration/flight. This is Oregon 
 State University's first time participating in this program. Participation this year will allow OSU
   to pursue more aerospace projects in the future. The RockSat-X program is based at Wallops Flight 
Facility and funded by the Space Grant Consortium. The rocket will be launched from Wallops and will
 expose students' experiments to space at the rocket's apogee. The program requires that students 
design and build an experimental or technical payload that makes use of microgravity or a space-like
environment.

\subsection{Limitations of Space Travel}
Currently space travel is limited by the need for constructing spacecraft on earth and launching into
orbit. Because of this, the scale of spacecraft is limited to what can be launched by current launch
 vehicles. The limited power of launch vehicles means that traveling beyond LEO has serious mass 
constraints. In order to travel further, we must circumvent this mass limitation. This can be 
solved by launching raw materials and parts and constructing the structure on orbit. 

The International Space Station is currently evaluating possible solutions to this problem. They are 
investigating the efficacy of 3D printing parts on orbit and of performing repairs with the help of 
a large robotic arm. The arm on the ISS has limited functionality because of its size.
 It has the ability to move astronauts around the outside of the ISS but is not capable of the detailed maneuvers that would be required for it performing repairs itself.


\section{Proposed Solution}
The Hephaestus payload shall be an assembly containing a mechanical arm capable of performing intricate maneuvers.
This project is a proof of concept for construction and repair on orbit using a mechanical arm. 
The ability to perform detailed maneuvers using an arm is critical for construction on orbit because
it replaces the current need for a spacewalk to perform the same task, saving time and money and decreasing risk.


\end{document}
